\documentclass[aspectratio=169]{beamer}
%\documentclass[aspectratio=43]{beamer}

\usepackage{graphicx}  % Required for including images
\usepackage{natbib}
\usepackage{booktabs} % Top and bottom rules for tables
\usepackage{amssymb,amsthm,amsmath}
\usepackage{exscale}
\usepackage{natbib}
\usepackage{tikz}
\usepackage{listings}
\usepackage{color}
\usepackage{animate}
\usepackage{bm}
\usepackage{etoolbox}

% Setup TikZ
\usepackage{tikz}
\usetikzlibrary{arrows}
\tikzstyle{block}=[draw opacity=0.7,line width=1.4cm]
% Setup hyperref
\usepackage{hyperref}
\hypersetup{colorlinks=true}
\hypersetup{citecolor=porange}
\hypersetup{urlcolor=porange!80!}
\hypersetup{linkcolor=porange}

\newtheorem{proposition}{Proposition}
\newtheorem{remark}{Remark}
\newtheorem{principle}{Principle}

%% Writing quarters
\newcommand{\wQ}[1]{{\textcolor{white}{Q#1}}}
\newcommand{\bQ}[1]{{Q#1}}

% Uncomment appropriate command to disable/enable hiding
%\newcommand{\mypause}{\pause}
\newcommand{\mypause}{}
\newcommand{\myb}[1]{{\color{blue} {#1}}}

%% Commonly used macros
\newcommand{\eqr}[1]{Eq.\thinspace(#1)}
\newcommand{\pfrac}[2]{\frac{\partial #1}{\partial #2}}
\newcommand{\pfracc}[2]{\frac{\partial^2 #1}{\partial #2^2}}
\newcommand{\pfraca}[1]{\frac{\partial}{\partial #1}}
\newcommand{\pfracb}[2]{\partial #1/\partial #2}
\newcommand{\pfracbb}[2]{\partial^2 #1/\partial #2^2}
\newcommand{\spfrac}[2]{{\partial_{#1}} {#2}}
\newcommand{\mvec}[1]{\mathbf{#1}}
\newcommand{\gvec}[1]{\boldsymbol{#1}}
\newcommand{\script}[1]{\mathpzc{#1}}
\newcommand{\eep}{\mvec{e}_\phi}
\newcommand{\eer}{\mvec{e}_r}
\newcommand{\eez}{\mvec{e}_z}
\newcommand{\iprod}[2]{\langle{#1}\rangle_{#2}}

\DeclareMathAlphabet{\mathpzc}{OT1}{pzc}{m}{it}

%% Autoscaled figures
\newcommand{\incfig}{\centering\includegraphics}
\setkeys{Gin}{width=0.9\linewidth,keepaspectratio}

%Make the items smaller
\newcommand{\cramplist}{
	\setlength{\itemsep}{0in}
	\setlength{\partopsep}{0in}
	\setlength{\topsep}{0in}}
\newcommand{\cramp}{\setlength{\parskip}{.5\parskip}}
\newcommand{\zapspace}{\topsep=0pt\partopsep=0pt\itemsep=0pt\parskip=0pt}

\newcommand{\backupbegin}{
   \newcounter{finalframe}
   \setcounter{finalframe}{\value{framenumber}}
}
\newcommand{\backupend}{
   \setcounter{framenumber}{\value{finalframe}}
}

\usetheme[bullet=circle,% Use circles instead of squares for bullets.
          titleline=true,% Show a line below the frame title.
          ]{Princeton}

\title[{\tt }]{Discontinuous Galerkin Schemes}%
\author[https://ast560.rtfd.io]%
{Ammar H. Hakim ({\tt ammar@princeton.edu}) \inst{1}}%

\institute[PPPL]
{ \inst{1} Princeton Plasma Physics Laboratory, Princeton, NJ %
}

\date[3/30/2021]{Princeton University, Course AST560, Spring 2021}

\begin{document}

\begin{frame}[plain]
  \titlepage
\end{frame}

\begin{frame}{Generalizing recovery: path to discontinous Galerkin schemes}
  \begin{itemize}
  \item In FV scheme we used \emph{cell-averages} to recover interface
    values for use in numerical fluxes%
    \mypause%
  \item What if we store more than just cell-averages? One can imagine
    in addition, mean-slope, mean-quadratic moments. Lead naturally to
    the concept of discontinous Galerkin schemes.
  \end{itemize}
  \mypause%
  The key connection is the concept of \emph{weak-equality}. Consider
  an interval $I$ and select a finite-dimensional function space on
  it, spanned by set of basis functions $\{ \psi_k \}$,
  $k=1,\ldots,N$. Choose an inner product, for example
  \begin{align*}
    (f, g) \equiv \int_I f(x) g(x) \thinspace dx.
  \end{align*}  
\end{frame}

\begin{frame}{Weak-equality}
  \small
  \begin{definition}[Weak equality]
    Two functions, $f$ and $g$ are said to be \emph{weakly equal} if
    \begin{align*}
      (\psi_k,f-g) = 0
    \end{align*}
    for all $k=1,\ldots,N$. We denote weak equality by
    \begin{align*}
      f \doteq g.
    \end{align*}
  \end{definition}  
  \begin{itemize}
  \item When we recovered polynomials across an interface in FV scheme
    we effectively choose a function space, $\{1 \}$ , with only
    \emph{one} basis function!
  \item In DG we can choose as many as we like: allows significant
    flexibility in designing accurate and compact schemes; suprisingly
    accurate for some problems.
  \end{itemize}
\end{frame}
  
\end{document}


\begin{frame}{}
\end{frame}

\begin{columns}
  
  \begin{column}{0.6\linewidth}
  \end{column}
  
  \begin{column}{0.4\linewidth}
    \includegraphics[width=\linewidth]{fig/Kinsey_2011_Pfus_vs_T.pdf}
  \end{column}
\end{columns}

% ----------------------------------------------------------------
