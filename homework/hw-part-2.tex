\documentclass[11pt]{article}
\usepackage[utf8]{inputenc}
\usepackage{times}
\usepackage{geometry}
\usepackage{tabularx}
\usepackage{hyperref}
\geometry{verbose,letterpaper,textheight=9in,textwidth=6.5in}
\usepackage{amsmath}
\usepackage{amsfonts}
\usepackage{amsthm}
\usepackage{wrapfig}%%enable enable 
\usepackage{graphicx}
\usepackage{amssymb}
\usepackage{color}
\usepackage{sidecap}
\usepackage{pdfpages}
\usepackage[font={footnotesize}]{caption}
%\usepackage[framemethod=tikz]{mdframed}
\def\bibfont{\footnotesize}
%\setlength{\bibsep}{0.0pt}
\usepackage{algorithm}
\usepackage[noend]{algpseudocode}

\newtheorem{proposition}{Proposition}
\newtheorem{theorem}{Theorem}
\newtheorem{lemma}{Lemma}
\newtheorem{remark}{Remark}
\newtheorem{definition}{Definition}
\newtheorem{principle}{Principle}

% Set the margins
\RequirePackage{geometry}
\geometry{letterpaper,includehead,tmargin=1in,bmargin=1in,lmargin=1in,rmargin=1in}
\geometry{headsep=2ex}

\newcommand{\grad}{\nabla}
\newcommand{\ignore}[1]{}  % ignore a section.

% Handy commands for notes and 'to do' items 
\newcommand{\todo}[1]{{\color{red}\bf [TODO: #1]}}
\newcommand{\note}[2]{{\color{blue} \textsf{[#1] #2}}}
\newcommand\tocite[1]{{[CITE: #1]}}

% For a clean copy, ignore these:
%\newcommand{\todo}[1]{}
%\newcommand{\note}[2]{}
%\newcommand\tocite[1]{}

%% Writing quarters
\newcommand{\wQ}[1]{{\textcolor{white}{Q#1}}}
\newcommand{\bQ}[1]{{Q#1}}

\DeclareMathAlphabet{\mathpzc}{OT1}{pzc}{m}{it}

%% Autoscaled figures
\newcommand{\incfig}{\centering\includegraphics}
\setkeys{Gin}{width=0.9\linewidth,keepaspectratio}

%% Commonly used macros
\newcommand{\eqr}[1]{Eq.\thinspace(#1)}
\newcommand{\pfrac}[2]{\frac{\partial #1}{\partial #2}}
% \newcommand{\pfracc}[2]{\frac{\partial^2 #1}{\partial #2^2}}
\newcommand{\pfraca}[1]{\frac{\partial}{\partial #1}}
\newcommand{\pfracb}[2]{\partial #1/\partial #2}
% \newcommand{\pfracbb}[2]{\partial^2 #1/\partial #2^2}
% \newcommand{\spfrac}[2]{{\partial_{#1}} {#2}}
% \newcommand{\mvec}[1]{\mathbf{#1}}
\newcommand{\gvec}[1]{\boldsymbol{#1}}
% \newcommand{\script}[1]{\mathpzc{#1}}
% \newcommand{\gke}{{\tt Gkeyll}}
\newcommand{\gcs}{\nabla_{\mvec{x}}}
\newcommand{\gvs}{\nabla_{\mvec{v}}}

% Other stuff
%\usepackage{units}
\usepackage{graphicx}
\usepackage{color}
\usepackage{hyperref}

\usepackage{tabularx}
\usepackage{mdwlist} % enumerate*, itemize* squeeze vertical space

\newcommand{\comment}[1]{\textit{\textcolor{red}{#1}}}
\renewcommand{\comment}[1]{}

% Turn off red comments by uncommenting the following line:
%\renewcommand{\comment}[1]{}

%% Autoscaled figures
%\newcommand{\incfig}{\centering\includegraphics}
\setkeys{Gin}{width=0.9\linewidth,keepaspectratio}
\newcommand{\gke}{{\tt Gkeyll}}
%\newcommand{\gke}{{\textsc{Gkeyll}}}
\newcommand{\mvec}[1]{\mathbf{#1}}

%Make the items smaller
\newcommand{\cramplist}{
	\setlength{\itemsep}{0in}
	\setlength{\partopsep}{0in}
	\setlength{\topsep}{0in}}
\newcommand{\cramp}{\setlength{\parskip}{.5\parskip}}
\newcommand{\zapspace}{\topsep=0pt\partopsep=0pt\itemsep=0pt\parskip=0pt}

\newcounter{probnum}
\setcounter{probnum}{1}

\title{Hyperbolic PDEs and Finite-Volume Methods. Homework Set I}%
\author{Ammar Hakim}%
\date{}%

\begin{document}
\maketitle

\section*{Summary}

Intuitively, hyperbolic PDEs describe phenomena that propagate with
finite speed. One way to check this is to \emph{linearize} the
equations around a uniform background and compute the dispersion
relation for propagation of small amplitude waves. A system linearized
$p$ hyperbolic equations will have dispersion relations of the form
$\omega^p = c_0^p k$ where $c_0^p$ are the speeds computed from the
equilibrium quantities. Note that the phase and group velocities for
linear waves are the same for hyperbolic systems.

\begin{definition}[Hyperbolic PDEs]
  Consider a system of conservation laws written as
  \begin{align}
    \pfrac{\mvec{Q}}{t} + \pfrac{\mvec{F}}{x} = 0. \label{eq:hyp-law}
  \end{align}
  where $\mvec{Q}$ is a vector of conserved quantities and
  $\mvec{F}(\mvec{Q})$ is a vector of fluxes. This system is called
  \emph{hyperbolic} if the flux Jacobian
  \begin{align*}
    \mvec{A} \equiv \pfrac{\mvec{F}}{\mvec{Q}}
  \end{align*}
  has \emph{real eigenvalues} and a \emph{complete set of linearly
    independent} eigenvectors. A hyperbolic system is called
  \emph{strictly hyperbolic} if all eigenvalues are distinct.
\end{definition}

To derive eigensystem it is sometimes easier to work in
non-conservative (quasi-linear) form of equations. Start with
\begin{align*}
  \pfrac{\mvec{Q}}{t} + \pfrac{\mvec{F}}{x} = 0.
\end{align*}
and introduce an invertible transform $\mvec{Q} = \varphi(\mvec{V})$
where $\mvec{V}$ are some other variables (for example: density,
velocity and pressure). Then the system converts to
\begin{align*}
  \pfrac{\mvec{V}}{t} +
  \underbrace{(\varphi^{\prime})^{-1} \mvec{A}\varphi^{\prime}}_{\mvec{B}}
  \pfrac{\mvec{V}}{x} = 0.
\end{align*}
Can easily show eigenvalues of $\mvec{A}$ are same as that of
$\mvec{B}$ and right eigenvectors can be computed from
$\varphi^{\prime} \mvec{r}_p$ and left eigenvectors from
$\mvec{l}_p (\varphi^{\prime})^{-1}$.

As the solution to hyperbolic PDEs can develop discontinuities we need
to instead consider a broader class of solutions than those supported
by the PDE (as derivatives are not defined at
discontinuities). Instead we must use the concept of
\emph{weak-solutions}, defined below.

\begin{definition}[Weak-solution]
  Let $\phi(x,t)$ be a compactly supported (i.e. zero outside some
  bounded region) smooth function (enough continuous
  derivatives). Then
  \begin{align}
    \int_0^\infty  \int_{-\infty}^\infty 
    \bigg[\pfrac{\phi}{t} \mvec{Q} + \pfrac{\phi}{x} \mvec{F}\bigg]\thinspace
    dx\thinspace dt
    =
    -
    \int_{-\infty}^\infty \phi(x,0) \mvec{Q}(x,0) dx.
    \label{eq:weak-sol}
  \end{align}
  is called the \emph{weak-form} of the conservation law
  \eqr{\ref{eq:hyp-law}}. A function $\mvec{Q}(x,t)$ is said to be a
  weak-solution if it satisfies the weak-form for all compact, smooth
  $\phi(x,t)$.
\end{definition}

Cast of characters: three characters appear in the problems below:
Ms. Fivo L.  Hacker, Mr. Fido Node and Prof Symplectico. One can guess
what point of view each of these characters represents. Also, for some
of these problem you may/will need to use a computer algebra system.

\section*{Problem \arabic{probnum}: Eigensystem of Euler Equations}
\stepcounter{probnum}

Consider the Euler equations in 1D conservative form
\begin{align*}
  \frac{\partial}{\partial{t}}
  \left[
    \begin{matrix}
      \rho \\
      \rho u \\
      E
    \end{matrix}
  \right]
  +
  \frac{\partial}{\partial{x}}
  \left[
    \begin{matrix}
      \rho u \\
      \rho u^2 + p \\
      (E+p)u
    \end{matrix}
  \right]
  =
  0
\end{align*}
where
\begin{align*}
  E = \frac{1}{2}\rho u^2 + \frac{p}{\gamma-1}.
\end{align*}
Show that these equations can also be written in the
\emph{quasi-linear} form
\begin{align*}
  \pfrac{\mvec{V}}{t} + \mvec{B}\pfrac{\mvec{V}}{x} = 0
\end{align*}
where $\mvec{V} = [\rho, u, p]^T$ and $\mvec{B}$ is a matrix you
should determine. Compute the eigenvalues and right eigenvectors of
$\mvec{B}$ and use them to compute the eigenvectors of the flux
Jacobian of the conservative system. Hence show that the Euler
equations are indeed hyperbolic as long as $p>0$ and $\rho > 0$.

\section*{Problem \arabic{probnum}: The effect of oscillatory source
  terms}
\stepcounter{probnum}

As discussed in class, source terms can significantly change the
physics contained in the homogenous system. As an example consider the
Euler equations with an oscillatory source as follows:
\begin{align*}
  \frac{\partial}{\partial{t}}
  \left[
    \begin{matrix}
      \rho \\
      \rho u \\
      \rho v \\      
      E
    \end{matrix}
  \right]
  +
  \frac{\partial}{\partial{x}}
  \left[
    \begin{matrix}
      \rho u \\
      \rho u^2 + p \\
      \rho u v \\
      (E+p)u
    \end{matrix}
  \right]
  =
  \left[
    \begin{matrix}
      0 \\
      \rho v\Omega \\
      -\rho u\Omega  \\
      0
    \end{matrix}
  \right]  
\end{align*}
where now
\begin{align*}
  E = \frac{1}{2}\rho (u^2+v^2) + \frac{p}{\gamma-1}
\end{align*}
and where $\Omega$ is a constant with units of inverse time. Linearize
this system around a uniform non-flowing ($u_0 = v_0 = 0$) background
$\rho_0, p_0$ and by considering solutions of the form
$e^{-i\omega t}e^{i kx }$ derive the dispersion relation
\begin{align*}
  \omega = \pm (k^2 c_{s0}^2 + \Omega^2)^{1/2}
\end{align*}
where $c_{s0} = \sqrt{\gamma p_0/\rho_0}$. Clearly, this system only
contains propagating but \emph{dispersive} waves, however it is not
hyperbolic as the dispersion relation is not \emph{linear}. Derive the
exact solution to the linearized system in terms of the perturbations
of the x-component of the velocity $u_1(x,t)$.

Finally, consider the initial velocity field
\begin{align*}
  u_1(x,0) = U_0 \sum_{n=0}^N \frac{i}{2n + 1} e^{i k_n x}
\end{align*}
for $x\in [0,1]$ and where $k_n = 2\pi (2n+1)$. For
$N\rightarrow \infty$ this represents a step function. Take $N=100$
and plot the exact solution of the perturbed density, $\rho_1(x,t)$,
at $t=1000$ for $\rho_0 = p_0 = 1$, $\gamma = 2$ and $\Omega =
10$. (If you make a movie of the exact solution in time you can see
the initial spectrum of waves disperse and form complex patterns. For
$\Omega = 0$ the initial condition will simply travel with sound speed
as in this case the system becomes hyperbolic and the waves have the
same group and phase velocities).

\section*{Problem \arabic{probnum}: (Optional, challenging) Eigenvalues of ideal MHD equations}
\stepcounter{probnum}

The ideal-MHD equations can be written in non-conservation law form
\begin{align*}
  \frac{\partial \rho}{\partial t}+ \mvec{u}\cdot\nabla\rho + \rho\nabla\cdot\mathbf{u}&=0 \\
  \frac{\partial \mathbf{u}}{\partial t}+\mathbf{u} \cdot \nabla
  \mathbf{u}
  +\frac{\nabla p}{\rho} &=
                           {\frac{1}{\mu_{0}\rho}(\nabla \times \mathbf{B}) \times \mathbf{B}} \\
    \frac{\partial p}{\partial t}+\mathbf{u} \cdot \nabla p+\gamma p
  \nabla \cdot \mathbf{u} &= 0 \\
  {\frac{\partial \mathbf{B}}{\partial t}-\nabla \times(\mathbf{u} \times \mathbf{B})} &= {0}
\end{align*}
with the constraint $\nabla\cdot\mvec{B} = 0$. Write this equation in
quasilinear form in 1D and compute the eigenvalues, showing they are
all real if $\rho>0$ and $p>0$.

\section*{Problem \arabic{probnum}: Shock solution for Burgers'
  equation}
\stepcounter{probnum}

In class we wrote down the solution to Burgers' equation for initial
condition $u(x,0) = u_l$ for $x<0$ and $u(x,0) > u_r$ for $x>0$ and
where $u_l > u_r$. That solution was:
\begin{align*}
  u(x,t) &= u_l \quad x < st \\
  u(x,t) &= u_r \quad x > st.
\end{align*}
Show that this indeed is a weak-solution of the Burgers' equation,
that is it satisfies \eqr{\ref{eq:weak-sol}}. Recall that the Burgers'
equation is a nonlinear scalar hyperbolic PDE:
\begin{align*}
    \pfrac{u}{t} + \frac{\partial}{\partial x}\bigg( \frac{1}{2} u^2
    \bigg) = 0.  
\end{align*}

\section*{Problem \arabic{probnum}: Reimann problem for Maxwell
  equations}
\stepcounter{probnum}

Find the exact solution to the the Reimann problem for Maxwell
equations in 1D:
\begin{align*}
  \frac{\partial }{\partial t}
  \left[
    \begin{matrix}
      E_y \\
      E_z \\
      B_y \\
      B_z
    \end{matrix}
  \right]
  +
  \frac{\partial }{\partial x}
  \left[
    \begin{matrix}
      c^2B_z \\
      -c^2B_y \\
      -E_z \\
      E_y
    \end{matrix}
  \right]
  =
  0.
\end{align*}
Note you do not actually need to compute the complete eigensystem of
these equations, though that is one way to solve the problem.

\section*{Problem \arabic{probnum}: $L_2$ norm and entropy}
\stepcounter{probnum}

Prof Symplectico points out to Fivo L. Hacker that the advection
equation for the distribution function $f(x,v,t)$ is a Hamiltonian
system
\begin{align*}
  \pfrac{f}{t} + v\pfrac{f}{x} = 0
\end{align*}
with the Hamiltonian $H = v^2/2$. Hence, it has infinite number of
Casimir invariants, including the $L_2$  norm
\begin{align*}
  \pfraca{t}\left(\frac{1}{2} f^2\right) + \pfraca{x}\left( \frac{1}{2} v f^2 \right) = 0.
\end{align*}
that must be preserved (even locally) by a good numerical
scheme. Prove that this expression is indeed true but point out, as
Fivo would, the key assumption used in deriving this expression and
why it may not be satisfied. Instead, to achieve stability she has
designed a scheme that \emph{decays} the $L_2$ norm, i.e.
\begin{align*}
  \frac{d}{dt}\int \frac{1}{2} f^2\thinspace dx\thinspace dv  \le 0.
\end{align*}
Using the identity $\ln(y) \le y-1$ show that Fivo's scheme, somewhat
reassuringly, \emph{increases} the entropy, i.e.
\begin{align*}
  \frac{d}{dt}\int -f \ln(f) \thinspace dx\thinspace dv  \ge 0.
\end{align*}

\section*{Problem \arabic{probnum}: Changing representations}
\stepcounter{probnum}

In class we stated the principle ``when studying or designing
numerical schemes {\bf never} confuse one solution representation for
another.'' Unfortunatley, Fivo L. Hacker's friend Fido Node has given
her initial conditions $f_i$, $i=1,\ldots,N$ evaluated at cell
centers. How will she convert this data to \emph{sufficiently high
  order} cell-average values to use in her third-order finite-volume
code?

Fivo also would like to compute the following integral
\begin{align*}
  E = \sum_j \int_{I_j} f g \thinspace dx
\end{align*}
As she learned in class this is not so simple as, in general,
$(f g)_i \neq f_i g_i$ (that is, the cell-average of the product is
not the product of cell-averages). How can this integral be computed
carefully such that the underlying order of her scheme (say third
order) can be preserved? Derive the stencil to do so.


\end{document}