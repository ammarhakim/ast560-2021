\documentclass[11pt]{article}
\usepackage[utf8]{inputenc}
\usepackage{times}
\usepackage{geometry}
\usepackage{tabularx}
\usepackage{hyperref}
\geometry{verbose,letterpaper,textheight=9in,textwidth=6.5in}
\usepackage{amsmath}
\usepackage{amsfonts}
\usepackage{amsthm}
\usepackage{wrapfig}%%enable enable 
\usepackage{graphicx}
\usepackage{amssymb}
\usepackage{color}
\usepackage{sidecap}
\usepackage{pdfpages}
\usepackage[font={footnotesize}]{caption}
%\usepackage[framemethod=tikz]{mdframed}
\def\bibfont{\footnotesize}
%\setlength{\bibsep}{0.0pt}
\usepackage{algorithm}
\usepackage[noend]{algpseudocode}

\newtheorem{proposition}{Proposition}
\newtheorem{theorem}{Theorem}
\newtheorem{lemma}{Lemma}
\newtheorem{remark}{Remark}
\newtheorem{definition}{Definition}
\newtheorem{principle}{Principle}

% Set the margins
\RequirePackage{geometry}
\geometry{letterpaper,includehead,tmargin=1in,bmargin=1in,lmargin=1in,rmargin=1in}
\geometry{headsep=2ex}

\newcommand{\grad}{\nabla}
\newcommand{\ignore}[1]{}  % ignore a section.

% Handy commands for notes and 'to do' items 
\newcommand{\todo}[1]{{\color{red}\bf [TODO: #1]}}
\newcommand{\note}[2]{{\color{blue} \textsf{[#1] #2}}}
\newcommand\tocite[1]{{[CITE: #1]}}

% For a clean copy, ignore these:
%\newcommand{\todo}[1]{}
%\newcommand{\note}[2]{}
%\newcommand\tocite[1]{}

%% Writing quarters
\newcommand{\wQ}[1]{{\textcolor{white}{Q#1}}}
\newcommand{\bQ}[1]{{Q#1}}

\DeclareMathAlphabet{\mathpzc}{OT1}{pzc}{m}{it}

%% Autoscaled figures
\newcommand{\incfig}{\centering\includegraphics}
\setkeys{Gin}{width=0.9\linewidth,keepaspectratio}

%% Commonly used macros
\newcommand{\eqr}[1]{Eq.\thinspace(#1)}
\newcommand{\pfrac}[2]{\frac{\partial #1}{\partial #2}}
% \newcommand{\pfracc}[2]{\frac{\partial^2 #1}{\partial #2^2}}
\newcommand{\pfraca}[1]{\frac{\partial}{\partial #1}}
\newcommand{\pfracb}[2]{\partial #1/\partial #2}
% \newcommand{\pfracbb}[2]{\partial^2 #1/\partial #2^2}
% \newcommand{\spfrac}[2]{{\partial_{#1}} {#2}}
% \newcommand{\mvec}[1]{\mathbf{#1}}
\newcommand{\gvec}[1]{\boldsymbol{#1}}
% \newcommand{\script}[1]{\mathpzc{#1}}
% \newcommand{\gke}{{\tt Gkeyll}}
\newcommand{\gcs}{\nabla_{\mvec{x}}}
\newcommand{\gvs}{\nabla_{\mvec{v}}}

% Other stuff
%\usepackage{units}
\usepackage{graphicx}
\usepackage{color}
\usepackage{hyperref}

\usepackage{tabularx}
\usepackage{mdwlist} % enumerate*, itemize* squeeze vertical space

\newcommand{\comment}[1]{\textit{\textcolor{red}{#1}}}
\renewcommand{\comment}[1]{}

% Turn off red comments by uncommenting the following line:
%\renewcommand{\comment}[1]{}

%% Autoscaled figures
%\newcommand{\incfig}{\centering\includegraphics}
\setkeys{Gin}{width=0.9\linewidth,keepaspectratio}
\newcommand{\gke}{{\tt Gkeyll}}
%\newcommand{\gke}{{\textsc{Gkeyll}}}
\newcommand{\mvec}[1]{\mathbf{#1}}

%Make the items smaller
\newcommand{\cramplist}{
	\setlength{\itemsep}{0in}
	\setlength{\partopsep}{0in}
	\setlength{\topsep}{0in}}
\newcommand{\cramp}{\setlength{\parskip}{.5\parskip}}
\newcommand{\zapspace}{\topsep=0pt\partopsep=0pt\itemsep=0pt\parskip=0pt}

\newcounter{probnum}
\setcounter{probnum}{1}

\title{Hyperbolic PDEs and Finite-Volume Methods. Homework Set II}%
\author{Ammar Hakim}%
\date{}%

\begin{document}
\maketitle

Use the notes on finite-volume methods posted on Blackboard/class
website. Write the code in any language you like. In general, for good
software practice you should use C or C++ and make sure you check all
your code into Github. Please do not forget that FV schemes evolve the
cell-average solution, specially when proving energy conservation and
checking convergence errors. Also remember that to get proper
convergence results you need to initialize the averages to
\emph{sufficiently high-order} using Gaussian quadrature.

\section*{Problem \arabic{probnum}: Third and fourth order Maxwell solver}
\stepcounter{probnum}

Write a FV solver for 1D Maxwell equations:
\begin{align*}
  \frac{\partial }{\partial t}
  \left[
    \begin{matrix}
      E_y \\
      B_z
    \end{matrix}
  \right]
  +
  \frac{\partial }{\partial x}
  \left[
    \begin{matrix}
      B_z \\
      E_y
    \end{matrix}
  \right]
  =
  0.
\end{align*}
Assume a \emph{periodic domain} and use an upwind biased third-order
recovery scheme, as well as a symmetric fourth-order scheme. Remember
that the upwinding needs to be applied to Riemann variables and not
$E_y$ and $B_z$ themselves. Show analytically that the fourth-order
symmetric scheme conserves total energy
\begin{align}
  \frac{d}{dt} \int_I (E_y^2 + B_z^2) \thinspace dx = 0.
\end{align}
where the integration is taken over the whole 1D domain. Test your
code against an exact solutions ...

\section*{Problem \arabic{probnum}: Third order solver for viscous
  Burgers' equations}
\stepcounter{probnum}

Write a FV solver for 1D viscous Burgers' equation
\begin{align}
  \pfrac{f}{t} + \pfraca{x} \bigg( \frac{1}{2} f^2 \bigg)
  = \nu \frac{\partial^2 f}{\partial x^2}.
\end{align}
Use third-order upwind biased scheme for the hyperbolic term and a
fourth-order symmetric scheme for the diffusion term.

\end{document}