\documentclass[11pt]{article}
\usepackage[utf8]{inputenc}
\usepackage{times}
\usepackage{geometry}
\usepackage{tabularx}
\usepackage{hyperref}
\geometry{verbose,letterpaper,textheight=9in,textwidth=6.5in}
\usepackage{amsmath}
\usepackage{amsfonts}
\usepackage{amsthm}
\usepackage{wrapfig}%%enable enable 
\usepackage{graphicx}
\usepackage{amssymb}
\usepackage{color}
\usepackage{sidecap}
\usepackage{pdfpages}
\usepackage[font={footnotesize}]{caption}
%\usepackage[framemethod=tikz]{mdframed}
\def\bibfont{\footnotesize}
%\setlength{\bibsep}{0.0pt}
\usepackage{algorithm}
\usepackage[noend]{algpseudocode}

\newtheorem{proposition}{Proposition}
\newtheorem{theorem}{Theorem}
\newtheorem{lemma}{Lemma}
\newtheorem{remark}{Remark}
\newtheorem{definition}{Definition}
\newtheorem{principle}{Principle}

% Set the margins
\RequirePackage{geometry}
\geometry{letterpaper,includehead,tmargin=1in,bmargin=1in,lmargin=1in,rmargin=1in}
\geometry{headsep=2ex}

\newcommand{\grad}{\nabla}
\newcommand{\ignore}[1]{}  % ignore a section.

% Handy commands for notes and 'to do' items 
\newcommand{\todo}[1]{{\color{red}\bf [TODO: #1]}}
\newcommand{\note}[2]{{\color{blue} \textsf{[#1] #2}}}
\newcommand\tocite[1]{{[CITE: #1]}}

% For a clean copy, ignore these:
%\newcommand{\todo}[1]{}
%\newcommand{\note}[2]{}
%\newcommand\tocite[1]{}

%% Writing quarters
\newcommand{\wQ}[1]{{\textcolor{white}{Q#1}}}
\newcommand{\bQ}[1]{{Q#1}}

\DeclareMathAlphabet{\mathpzc}{OT1}{pzc}{m}{it}

%% Autoscaled figures
\newcommand{\incfig}{\centering\includegraphics}
\setkeys{Gin}{width=0.9\linewidth,keepaspectratio}

%% Commonly used macros
\newcommand{\eqr}[1]{Eq.\thinspace(#1)}
\newcommand{\pfrac}[2]{\frac{\partial #1}{\partial #2}}
% \newcommand{\pfracc}[2]{\frac{\partial^2 #1}{\partial #2^2}}
\newcommand{\pfraca}[1]{\frac{\partial}{\partial #1}}
\newcommand{\pfracb}[2]{\partial #1/\partial #2}
% \newcommand{\pfracbb}[2]{\partial^2 #1/\partial #2^2}
% \newcommand{\spfrac}[2]{{\partial_{#1}} {#2}}
% \newcommand{\mvec}[1]{\mathbf{#1}}
\newcommand{\gvec}[1]{\boldsymbol{#1}}
% \newcommand{\script}[1]{\mathpzc{#1}}
% \newcommand{\gke}{{\tt Gkeyll}}
\newcommand{\gcs}{\nabla_{\mvec{x}}}
\newcommand{\gvs}{\nabla_{\mvec{v}}}

% Other stuff
%\usepackage{units}
\usepackage{graphicx}
\usepackage{color}
\usepackage{hyperref}

\usepackage{tabularx}
\usepackage{mdwlist} % enumerate*, itemize* squeeze vertical space

\newcommand{\comment}[1]{\textit{\textcolor{red}{#1}}}
\renewcommand{\comment}[1]{}

% Turn off red comments by uncommenting the following line:
%\renewcommand{\comment}[1]{}

%% Autoscaled figures
%\newcommand{\incfig}{\centering\includegraphics}
\setkeys{Gin}{width=0.9\linewidth,keepaspectratio}
\newcommand{\gke}{{\tt Gkeyll}}
%\newcommand{\gke}{{\textsc{Gkeyll}}}
\newcommand{\mvec}[1]{\mathbf{#1}}

%Make the items smaller
\newcommand{\cramplist}{
	\setlength{\itemsep}{0in}
	\setlength{\partopsep}{0in}
	\setlength{\topsep}{0in}}
\newcommand{\cramp}{\setlength{\parskip}{.5\parskip}}
\newcommand{\zapspace}{\topsep=0pt\partopsep=0pt\itemsep=0pt\parskip=0pt}

\title{Hyperbolic PDEs and Finite-Volume Methods. Homework Set I}%
\author{Ammar Hakim}%
\date{}%

\begin{document}
\maketitle

Intuitively, hyperbolic PDEs describe phenomena that propagate with
finite speed. One way to check this is to \emph{linearize} the
equations and compute the dispersion relation for propagation of small
amplitude waves. A system linearized $p$ hyperbolic equations will
have dispersion relations of the form $\omega^p = c_0^p k$ where
$c_0^p$ are the speeds computed from the equilibrium quantities. Note
that the phase and group velocities for linear waves are the same for
hyperbolic systems.

\begin{definition}[Hyperbolic PDEs]
  Consider a system of conservation laws written as
  \begin{align}
    \pfrac{\mvec{Q}}{t} + \pfrac{\mvec{F}}{x} = 0. \label{eq:hyp-law}
  \end{align}
  where $\mvec{Q}$ is a vector of conserved quantities and
  $\mvec{F}(\mvec{Q})$ is a vector of fluxes. This system is called
  \emph{hyperbolic} if the flux Jacobian
  \begin{align*}
    \mvec{A} \equiv \pfrac{\mvec{F}}{\mvec{Q}}
  \end{align*}
  has \emph{real eigenvalues} and a \emph{complete set of linearly
    independent} eigenvectors. A hyperbolic system is called
  \emph{strictly hyperbolic} if all eigenvalues are distinct.
\end{definition}

As the solution to hyperbolic PDEs can develop discontinuities we need
to instead consider a broader class of solutions than those supported
by the PDE (as derivatives are not defined at
discontinuities). Instead we must use the concept of
\emph{weak-solutions}, defined below.

\begin{definition}[Weak-solution]
  Let $\phi(x,t)$ be a compactly supported (i.e. zero outside some
  bounded region) smooth function (enough continuous
  derivatives). Then
  \begin{align*}
    \int_0^\infty  \int_{-\infty}^\infty 
    \bigg[\pfrac{\phi}{t} \mvec{Q} + \pfrac{\phi}{x} \mvec{F}\bigg]\thinspace
    dx\thinspace dt
    =
    -
    \int_{-\infty}^\infty \phi(x,0) \mvec{Q}(x,0) dx.
  \end{align*}
  is called the \emph{weak-form} of the conservation law
  \eqr{\ref{eq:hyp-law}}. A function $\mvec{Q}(x,t)$ is said to be a
  weak-solution if it satisfies the weak-form for all compact, smooth
  $\phi(x,t)$.
\end{definition}
\noindent You should be able to derive this expression starting from
\eqr{\ref{eq:hyp-law}}.

\end{document}