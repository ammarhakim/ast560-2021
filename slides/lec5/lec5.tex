\documentclass[aspectratio=169]{beamer}
%\documentclass[aspectratio=43]{beamer}

\usepackage{graphicx}  % Required for including images
\usepackage{natbib}
\usepackage{booktabs} % Top and bottom rules for tables
\usepackage{amssymb,amsthm,amsmath}
\usepackage{exscale}
\usepackage{natbib}
\usepackage{tikz}
\usepackage{listings}
\usepackage{color}
\usepackage{animate}
\usepackage{bm}
\usepackage{etoolbox}

% Setup TikZ
\usepackage{tikz}
\usetikzlibrary{arrows}
\tikzstyle{block}=[draw opacity=0.7,line width=1.4cm]
% Setup hyperref
\usepackage{hyperref}
\hypersetup{colorlinks=true}
\hypersetup{citecolor=porange}
\hypersetup{urlcolor=porange!80!}
\hypersetup{linkcolor=porange}

\newtheorem{proposition}{Proposition}
\newtheorem{remark}{Remark}
\newtheorem{principle}{Principle}

%% Writing quarters
\newcommand{\wQ}[1]{{\textcolor{white}{Q#1}}}
\newcommand{\bQ}[1]{{Q#1}}

% Uncomment appropriate command to disable/enable hiding
\newcommand{\mypause}{\pause}
%\newcommand{\mypause}{}
\newcommand{\myb}[1]{{\color{blue} {#1}}}

%% Commonly used macros
\newcommand{\eqr}[1]{Eq.\thinspace(#1)}
\newcommand{\pfrac}[2]{\frac{\partial #1}{\partial #2}}
\newcommand{\pfracc}[2]{\frac{\partial^2 #1}{\partial #2^2}}
\newcommand{\pfraca}[1]{\frac{\partial}{\partial #1}}
\newcommand{\pfracb}[2]{\partial #1/\partial #2}
\newcommand{\pfracbb}[2]{\partial^2 #1/\partial #2^2}
\newcommand{\spfrac}[2]{{\partial_{#1}} {#2}}
\newcommand{\mvec}[1]{\mathbf{#1}}
\newcommand{\gvec}[1]{\boldsymbol{#1}}
\newcommand{\script}[1]{\mathpzc{#1}}
\newcommand{\eep}{\mvec{e}_\phi}
\newcommand{\eer}{\mvec{e}_r}
\newcommand{\eez}{\mvec{e}_z}
\newcommand{\iprod}[2]{\langle{#1}\rangle_{#2}}

\DeclareMathAlphabet{\mathpzc}{OT1}{pzc}{m}{it}

%% Autoscaled figures
\newcommand{\incfig}{\centering\includegraphics}
\setkeys{Gin}{width=0.9\linewidth,keepaspectratio}

%Make the items smaller
\newcommand{\cramplist}{
	\setlength{\itemsep}{0in}
	\setlength{\partopsep}{0in}
	\setlength{\topsep}{0in}}
\newcommand{\cramp}{\setlength{\parskip}{.5\parskip}}
\newcommand{\zapspace}{\topsep=0pt\partopsep=0pt\itemsep=0pt\parskip=0pt}

\newcommand{\backupbegin}{
   \newcounter{finalframe}
   \setcounter{finalframe}{\value{framenumber}}
}
\newcommand{\backupend}{
   \setcounter{framenumber}{\value{finalframe}}
}

\usetheme[bullet=circle,% Use circles instead of squares for bullets.
          titleline=true,% Show a line below the frame title.
          ]{Princeton}

\title[{\tt }]{From Finite-Volume Methods to Discontinuous Galerkin Schemes}%
\author[https://ast560.rtfd.io]%
{Ammar H. Hakim ({\tt ammar@princeton.edu}) \inst{1}}%

\institute[PPPL]
{ \inst{1} Princeton Plasma Physics Laboratory, Princeton, NJ %
}

\date[3/25/2021]{Princeton University, Course AST560, Spring 2021}

\begin{document}

\begin{frame}[plain]
  \titlepage
\end{frame}



\begin{frame}{Godunov's Theorem}
  \small%
  \begin{itemize}
  \item A very important theorem proved by Godunov is that there is {
      \bf no} \emph{linear scheme} that is ``monotonicity preserving''
    (no new maxima/minima created) and {\bf higher than first-order
      accurate}!  \mypause%
  \item Consider a general scheme for advection equation
    \begin{align*}
      f_j^{n+1} = \sum_k c_k f_{j+k}^n.
    \end{align*}
    The discrete slope then is
    \begin{align*}
      f_{j+1}^{n+1} - f_j^{n+1} = \sum_k c_k \left( f_{j+k+1}^n - f_{j+k}^n \right).
    \end{align*}
    Assume that all $f_{j+1}^n - f_j^n > 0$. To maintain monotonicity
    at next time-step hence one must have all $c_k \ge 0$.
  \end{itemize}
\end{frame}

\begin{frame}{Godunov's Theorem}
  \small%
  \begin{itemize}
  \item First order upwind scheme:
    \begin{align*}
      f_j^{n+1} = f_j^n -\frac{\Delta t}{\Delta x}(f_j^n - f_{j-1}^n)
    \end{align*}
    this satisfies monotonicity as long as
    ${\Delta t}/{\Delta x} \le 1$.%
    \mypause%
  \item Second order symmetric scheme
    \begin{align*}
      f_j^{n+1} = f_j^n -\frac{\Delta t}{2 \Delta x}(f_{j+1}^n - f_{j-1}^n)
    \end{align*}
    clearly this does not satisfy the condition of monotonicity.
    \mypause%
  \item In general condition on Taylor series to ensure atleast
    second-order accuracy shows that at least \emph{one} of the $c_k$s
    must be negative. Hence, by contradiction, \emph{no such scheme
      exists}!
  \end{itemize}
\end{frame}

\begin{frame}{Godunov's Theorem: Unfortunate Consequences and Workarounds}
  \small%
  \begin{itemize}
  \item Godunov's Theorem is highly distressing: accurate
    discretization seems to preclude a scheme free from monotonicity
    violations
    \mypause%
  \item One way around is to start with a linear scheme that is very
    accurate and then add some local diffusion to it to control the
    monotonicity.%
    \mypause%
  \item However, Godunov's theorem shows that this ``diffusion'' must
    be dependent on the local solution itself and can't be fixed
    \emph{a priori}. This means a {\bf monotonicity preserving scheme
      must be nonlinear}, even for linear hyperbolic equations.%
    \mypause%
  \item Leads to the concept of \emph{nonlinear limiters} that control
    the monotonicity violations (adding diffusion to high-$k$ modes).
    No free lunch: limiters must diffuse high-$k$ modes but this will
    inevitably lead to issues like inability to capture, for example,
    high-$k$ turbulence spectra correctly without huge grids.
  \item Major research project: interaction of shocks, boundary layers
    and turbulence in high-Reynolds number flows.
  \end{itemize}
\end{frame}

\begin{frame}{Godunov's Theorem: Workarounds}
  \begin{itemize}
  \item To prove Godunov's Theorem we assumed that the solution at the
    next time-step was a \emph{linear} combination of the old
    time-step. However, this leads to monotonicity violations.%
    \mypause%
  \item Hence, there may be hope that a \emph{nonlinear} combination
    might allow constructing a monotonic (shock-capturing) scheme.%
    \mypause%
  \item Several ways to do this: we know that first-order upwind
    preserves monotonicity. Hence, perhaps combine a high-order
    recovery with first-order upwind in ``non-smooth'' regions.
    \mypause%
  \item Use a \emph{nonlinear recovery} for computing the interface
    value. This leads to the concept of \emph{limiters}.%
    \mypause%
  \item Use symmetric diffusion free scheme and add controlled
    \emph{hyper-diffusion} depending on \emph{local} solution.
  \end{itemize}

\end{frame}

\begin{frame}{Diffusion, Hyperdiffusion}
  Consider adding a diffusion term to the RHS of the advection
  equation
  \begin{align*}
    \pfrac{f}{t} + \pfrac{f}{x} = \nu_2\frac{\partial^2 f}{\partial x^2}
  \end{align*}
  The diffusion term will damp modes as $-\nu_2 k^2$. Unfortunately,
  even long-wavelength modes will be damped: Not good!
  \mypause%
  How about
  \begin{align*}
    \pfrac{f}{t} + \pfrac{f}{x} = -\nu_4\frac{\partial^4 f}{\partial x^4}
  \end{align*}
  This will damp modes as $-\nu_4 k^4$. Even higher-order
  hyper-diffusion can be added.
\end{frame}

\begin{frame}{Diffusion, Hyperdiffusion}
  To discretize the (hyper)-diffusion use \emph{symmetric recovery}
  based scheme:
  \begin{align*}
    \pfrac{f_j}{t} + \frac{f_{j+1/2}-f_{j-1/2}}{\Delta x}
    =
    \frac{\nu_2}{\Delta x}
    \left(
    \pfrac{f_{j+1/2}}{x} - \pfrac{f_{j-1/2}}{x}
    \right)
  \end{align*}
  For example, use 4-cell symmetric recovery across each interface and
  use its \emph{first-derivative} to compute gradients at each edge.
  \mypause%
  \begin{itemize}
  \item The same recovery polynomial need not be used to compute the
    hyperbolic terms and diffusion term: use upwind biased for
    hyperbolic, for example, and symmetric recovery for
    (hyper)-diffusion.
  \item For hyper-diffusion a \emph{wider} stencil may be needed if
    order is to be maintained.
  \end{itemize}
\end{frame}  

\begin{frame}{Numerical Flux Function: Lax flux}
  \small%
  \begin{itemize}
  \item A good choice of the numerical flux function is the
    \emph{local Lax} flux:
    \begin{align*}
      \mvec{G}(\mvec{Q}_L,\mvec{Q}_R) = \frac{1}{2}
      \left( \mvec{F}(\mvec{Q}_L) + \mvec{F}(\mvec{Q}_R) \right)
      - \frac{|\lambda|}{2}( \mvec{Q}_R - \mvec{Q}_L )
    \end{align*}
    where $|\lambda|$ is an estimate of the (absolute) maximum of all
    eigenvalues at the interface.%
    \mypause%
  \item   For advection equation this becomes
    \begin{align*}
      G(f_L,f_R) = \frac{1}{2} a ( f_L + f_R ) - \frac{|a|}{2}( f_R - f_L )
    \end{align*}
    This works for either sign of advection speed $a$, automatically
    giving upwinding.%
    \mypause%
  \item Note $|\lambda|$ is only a local (to the interface)
    \emph{estimate}. You can use a global estimate too: orginal
    formulation by Peter Lax (``Lax fluxes'').
  \end{itemize}

\end{frame}

\begin{frame}{Numerical Flux Function: Systems of equations}
  \small%
  \begin{itemize}
  \item Lax flux is a good ``first'' flux to use. However, notice it
    only takes into account a \emph{single} piece of information:
    maximum eigenvalue.%
    \mypause%
  \item For a \emph{linear system} of equations (Maxwell equation) or
    \emph{locally linearized} nonlinear system we can instead do
    \begin{align*}
      G(Q_R,Q_L) = \frac{1}{2}\big(F(Q_R)+F(Q_L)\big) - \frac{1}{2}(A^+\Delta Q_{R,L} - A^-\Delta Q_{R,L})      
    \end{align*}
    where the \emph{fluctuations} $A^\pm\Delta Q$ are defined as
    \begin{align*}
      A^\pm\Delta Q_{R,L} \equiv \sum_p r^p \lambda^\pm_p (w_R^p-w_L^p) = \sum_p r^p \lambda^\pm_p l^p(Q_R-Q_L).
    \end{align*}
    where $\lambda_p^+ = \max(\lambda_p,0)$ and
    $\lambda_p^- = \min(\lambda_p,0)$.%
    \mypause%
  \item Additional care is needed for nonlinear equations like Euler
    or ideal MHD equations.
  \end{itemize}
\end{frame}

\begin{frame}{Numerical Flux Function: Nonlinear equations}
  \begin{itemize}
  \item For nonlinear scalar Burgers' equation ($F(f) = f^2/2$) we can
    use Lax-fluxes
    \begin{align*}
      G(f_L,f_R) = \frac{1}{4} ( f_L^2 + f_R^2 ) - \frac{|s|}{2}( f_R - f_L )
    \end{align*}
    where now we compute the speed $s = (f_R+f_L)/2$. Note that the
    speed $s$ is that of a shock with left/right values $f_R$ and
    $f_L$.%
    \mypause%
  \item For more complicated system of equations (Euler, ideal MHD)
    one can use Lax fluxes (good first choice). Or, use a Roe solver
    (JCP, {\bf 43}, 357-372 (1981)).%
    \mypause%
  \item Roe solvers work by computing the flux Jacobian using suitable
    ``Roe averages'' and using these to compute eigenvalues and
    left/right eigenvectors and using the flux function for linear
    system of equations.
  \end{itemize}

\end{frame}

\begin{frame}{Roe averaging and Roe solver for numerical flux}
  \small%
  \begin{itemize}
  \item The Roe solver works by local linearization at an interface in
    which the nonlinear equation is written as
    \begin{align*}
      \pfrac{\mvec{Q}}{t} + \mvec{A}\pfrac{\mvec{Q}}{x} = 0
    \end{align*}
    where $\mvec{A} = \mvec{A}(\mvec{Q}_L,\mvec{Q}_R)$ is the flux
    Jacobian computed using left/right states at that interface.%
    \mypause
  \item For conservation one must ensure the \emph{flux-jump}
    condition
    \begin{align*}
      \mvec{A}(\mvec{Q}_L,\mvec{Q}_R)(\mvec{Q}_L-\mvec{Q}_R)
      =
      \mvec{F}(\mvec{Q}_L)-\mvec{F}(\mvec{Q}_R).
    \end{align*}
    This puts serious constraints on how $\mvec{A}$ is
    constructed. Roe realized a special way in which this could be
    done for Euler equations. Probably the most important advance in
    numerical methods for shock-dominanted/supersonic flow problems.%
    \mypause%
  \item Vast (majority?) number of fluid and MHD codes in CFD,
    astrophysics, (general-)relativistic hydro, use Roe solver. Huge
    number of variants and ``fixes''.
  \end{itemize}
\end{frame}

\begin{frame}{Generalizing recovery: path to discontinous Galerkin schemes}
  \begin{itemize}
  \item In FV scheme we used \emph{cell-averages} to recover interface
    values for use in numerical fluxes%
    \mypause%
  \item What if we store more than just cell-averages? One can imagine
    in addition, mean-slope, mean-quadratic moments. Lead naturally to
    the concept of discontinous Galerkin schemes.
  \end{itemize}
  \mypause%
  The key connection is the concept of \emph{weak-equality}. Consider
  an interval $I$ and select a finite-dimensional function space on
  it, spanned by basis functions $\psi_k$, $k=1,\ldots,N$. Choose an
  inner product, for example
  \begin{align*}
    (f, g) \equiv \int_I f(x) g(x) \thinspace dx.
  \end{align*}  
\end{frame}

\begin{frame}{Weak-equality}
  \small
  \begin{definition}[Weak equality]
    Two functions, $f$ and $g$ are said to be \emph{weakly equal} if
    \begin{align*}
      (\psi_k,f-g) = 0
    \end{align*}
    for all $k=1,\ldots,N$. We denote weak equality by
    \begin{align*}
      f \doteq g.
    \end{align*}
  \end{definition}  
  \begin{itemize}
  \item When we recovered polynomials across an interface in FV scheme
    we effectively choose a function space with only \emph{one} basis
    function!
  \item In DG we can choose as many as we like: allows significant
    flexibility in designing accurate and compact schemes; suprisingly
    accurate for some problems.
  \end{itemize}
\end{frame}
  
\end{document}


\begin{frame}{}
\end{frame}

\begin{columns}
  
  \begin{column}{0.6\linewidth}
  \end{column}
  
  \begin{column}{0.4\linewidth}
    \includegraphics[width=\linewidth]{fig/Kinsey_2011_Pfus_vs_T.pdf}
  \end{column}
\end{columns}

% ----------------------------------------------------------------
