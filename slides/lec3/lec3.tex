\documentclass[aspectratio=169]{beamer}
%\documentclass[aspectratio=43]{beamer}

\usepackage{graphicx}  % Required for including images
\usepackage{natbib}
\usepackage{booktabs} % Top and bottom rules for tables
\usepackage{amssymb,amsthm,amsmath}
\usepackage{exscale}
\usepackage{natbib}
\usepackage{tikz}
\usepackage{listings}
\usepackage{color}
\usepackage{animate}
\usepackage{bm}
\usepackage{etoolbox}

% Setup TikZ
\usepackage{tikz}
\usetikzlibrary{arrows}
\tikzstyle{block}=[draw opacity=0.7,line width=1.4cm]
% Setup hyperref
\usepackage{hyperref}
\hypersetup{colorlinks=true}
\hypersetup{citecolor=porange}
\hypersetup{urlcolor=porange!80!}
\hypersetup{linkcolor=porange}

\newtheorem{proposition}{Proposition}
\newtheorem{remark}{Remark}
\newtheorem{principle}{Principle}

%% Writing quarters
\newcommand{\wQ}[1]{{\textcolor{white}{Q#1}}}
\newcommand{\bQ}[1]{{Q#1}}

% Uncomment appropriate command to disable/enable hiding
%\newcommand{\mypause}{\pause}
\newcommand{\mypause}{}
\newcommand{\myb}[1]{{\color{blue} {#1}}}

%% Commonly used macros
\newcommand{\eqr}[1]{Eq.\thinspace(#1)}
\newcommand{\pfrac}[2]{\frac{\partial #1}{\partial #2}}
\newcommand{\pfracc}[2]{\frac{\partial^2 #1}{\partial #2^2}}
\newcommand{\pfraca}[1]{\frac{\partial}{\partial #1}}
\newcommand{\pfracb}[2]{\partial #1/\partial #2}
\newcommand{\pfracbb}[2]{\partial^2 #1/\partial #2^2}
\newcommand{\spfrac}[2]{{\partial_{#1}} {#2}}
\newcommand{\mvec}[1]{\mathbf{#1}}
\newcommand{\gvec}[1]{\boldsymbol{#1}}
\newcommand{\script}[1]{\mathpzc{#1}}
\newcommand{\eep}{\mvec{e}_\phi}
\newcommand{\eer}{\mvec{e}_r}
\newcommand{\eez}{\mvec{e}_z}
\newcommand{\iprod}[2]{\langle{#1}\rangle_{#2}}

\DeclareMathAlphabet{\mathpzc}{OT1}{pzc}{m}{it}

%% Autoscaled figures
\newcommand{\incfig}{\centering\includegraphics}
\setkeys{Gin}{width=0.9\linewidth,keepaspectratio}

%Make the items smaller
\newcommand{\cramplist}{
	\setlength{\itemsep}{0in}
	\setlength{\partopsep}{0in}
	\setlength{\topsep}{0in}}
\newcommand{\cramp}{\setlength{\parskip}{.5\parskip}}
\newcommand{\zapspace}{\topsep=0pt\partopsep=0pt\itemsep=0pt\parskip=0pt}

\newcommand{\backupbegin}{
   \newcounter{finalframe}
   \setcounter{finalframe}{\value{framenumber}}
}
\newcommand{\backupend}{
   \setcounter{framenumber}{\value{finalframe}}
}

\usetheme[bullet=circle,% Use circles instead of squares for bullets.
          titleline=true,% Show a line below the frame title.
          ]{Princeton}

\title[{\tt }]{Hyperbolic PDEs and Finite-Volume Methods III}%
\author[https://ast560.rtfd.io]%
{Ammar H. Hakim ({\tt ammar@princeton.edu}) \inst{1}}%

\institute[PPPL]
{ \inst{1} Princeton Plasma Physics Laboratory, Princeton, NJ %
}

\date[3/18/2021]{Princeton University, Course AST560, Spring 2021}

\begin{document}

\begin{frame}[plain]
  \titlepage
\end{frame}

\begin{frame}{The Riemann Problem for hyperbolic PDEs}
  \small%
  The Riemann problem is a class of \emph{initial value} problems for
  a hyperbolic PDE
  \begin{align*}
    \pfrac{\mvec{Q}}{t} + \pfrac{\mvec{F}}{x} = 0.
  \end{align*}
  on $x\in[-\infty,\infty]$ with initial conditions
  \begin{align*}
    \mvec{Q}(x,0) = \mvec{Q}_R \quad x>0 \\
    \mvec{Q}(x,0) = \mvec{Q}_L \quad x<0    
  \end{align*}
  where $\mvec{Q}_{L,R}$ are \emph{constant} initial states.%
  \begin{itemize}
  \item Fundamental mathematical problem in theory of hyperbolic PDEs:
    brings out the key structure of the nonlinear solutions of the
    system.
  \item For some important systems like (relativisitic) Euler
    equations, ideal MHD the Riemann problem can be solved
    \emph{exactly} (modulo some nonlinear root-finding).
  \item Good test for shock-capturing schemes as it tests ability to
    capture discontinuities and complex non-linear phenomena.
  \end{itemize}
\end{frame}  

\begin{frame}{Essence of the finite-volume method}
  \small%
  Consider a PDE of the form (non necessarily hyperbolic)
  \begin{align*}
    \pfrac{\mvec{Q}}{t} + \pfrac{\mvec{F}}{x} = 0.    
  \end{align*}
  Now make a grid with cells $I_j = [x_{j-1},x_{j+1/2}]$ and
  $\Delta x = x_{j+1/2} - x_{j-1/2}$. The finite-volume method
  \emph{usually} evolves the cell-averages of the solution:
  \begin{align*}
    \pfrac{\mvec{Q}_j}{t} + \frac{\mvec{F}_{j+1/2} -\mvec{F}_{j-1/2} }{\Delta x} = 0
  \end{align*}
  where
  \begin{align*}
    \mvec{Q}_j(t) \equiv \frac{1}{\Delta x}\int_{I_j} \mvec{Q}(x,t) \thinspace dx
  \end{align*}
  are the \emph{cell-averages} and
  \begin{align*}
    \mvec{F}_{j\pm 1/2} \equiv \mvec{F}(\mvec{Q}_{j\pm 1/2})
  \end{align*}
  are at cell interfaces.

\end{frame}

\begin{frame}{Essence of the finite-volume method}
  \small%
  The finite-volume method \emph{usually} evolves the cell-averages of
  the solution:
  \begin{align*}
    \pfrac{\mvec{Q}_j}{t} + \frac{\mvec{F}_{j+1/2} -\mvec{F}_{j-1/2} }{\Delta x} = 0
  \end{align*}
  This equation is an \emph{exact} evolution equation for the
  cell-averages.\mypause%
  However, notice that
  \begin{itemize}
  \item We only know cell-averages $\mvec{Q}_{j}$ in each cell; we
    \emph{do not} know the \emph{cell-edge} values
    $\mvec{Q}_{j\pm 1/2}$ needed to compute the flux
    $\mvec{F}_{j\pm 1/2}$.%
    \mypause%
  \item The finite-volume method consists of determining these
    \emph{edge values} and \emph{constructing a numerical-flux} so the
    cell-averages can be updated.
  \item Time-stepping can be done with a ODE solver (method-of-lines)
    or using a \emph{single-step} method (fully discrete scheme).
  \end{itemize}
  
\end{frame}

\begin{frame}{Cell-averages v/s cell-center values}
  \small%
  \begin{itemize}
  \item Typically, finite-volume schemes evolve the cell-average
    values; finite-difference schemes evolve nodal values.%
    \mypause%
  \item For some low-order (first and some second-order) schemes the
    \emph{forms} of the scheme may look superficially the
    same. However, this is not true in general and one must \emph{very
      carefully} distinguish between cell-average and point-wise
    values. Otherwise incorrect schemes can result that ``look okay''
    but do not achieve full accuracy.%
    \mypause%
  \item What we evolve (cell-average, nodal values or in DG moments or
    interior node values) is called the \emph{solution
      representation}.
  \end{itemize}
  \begin{block}{Remember Your Representation}
    When studying or designing numerical schemes {\bf never} confuse
    one solution representation for another.
  \end{block}  
\end{frame}

\begin{frame}{Finite-Volume method computes \emph{mean} of flux gradient}
  To derive the basic form of the scheme we did
  \begin{align*}
    \frac{1}{\Delta x}\int_{I_j} \pfrac{F}{x} \thinspace dx =
    \frac{F_{j+1/2}-F_{j-1/2}}{\Delta x}.
  \end{align*}
  \begin{itemize}
  \item Notice that the left-hand side is the \emph{mean} of the flux
    gradient in the cell $I_j$
  \item Hence, in effect, the FV scheme is computing the \emph{mean}
    of the flux gradient and not the flux gradient itself. This is
    then used to update \emph{cell-average} of the solution.
  \item This is important to remember when computing source terms;
    making plots or computing diagnostics. (Remember Your
    Representation!).
  \end{itemize}
  
\end{frame}

\begin{frame}{Example: How to compute mean of \emph{product} of
    values?}
  \begin{itemize}
  \item Given cell-average values $Q_j$ and $V_j$ how can you compute
    cell-average value $(QV)_j$?%
    \mypause%
  \item Clearly, $(Q V)_j$ is not the same as $Q_j V_j$.%
    \mypause%
  \item In general, depending on the order of the scheme one has to
    \emph{recover} $Q(x)$ and $V(x)$ to sufficiently high order in a
    cell, multiply them and then compute the average of the
    product. Potential complications when solutions are not smooth
    enough.
  \item Almost never done! However, it may be important when trying to
    extract delicate information from simulations like turbulence
    spectra etc.
  \end{itemize}
\end{frame}

\begin{frame}{Essence of the finite-volume method}
  \footnotesize
  \begin{columns}
  
    \begin{column}{0.5\linewidth}
      Instead of computing one edge value we will compute \emph{two}
      values: one the left and one on right of cell-edge. We will next
      define a \emph{numerical flux function}
      \begin{align*}
        \mvec{G} = \mvec{G}(\mvec{Q}^{-}_{j+1/2},\mvec{Q}^{+}_{j+1/2})
      \end{align*}
      with \emph{consistency} condition
      \begin{align*}
        \lim_{\mvec{Q}_{L,R}\rightarrow Q} \mvec{G}(\mvec{Q}_L,\mvec{Q}_R) = \mvec{F}(\mvec{Q})
      \end{align*}
    \end{column}
  
    \begin{column}{0.5\linewidth}
      \begin{figure}    
        \setkeys{Gin}{width=0.7\linewidth,keepaspectratio}
        \incfig{FV-1D-grid.pdf}
      \end{figure}    
    \end{column}
  \end{columns}
  \mypause%
  In terms of the numerical flux function the FV update formula
  becomes
  \begin{align*}
    \pfrac{\mvec{Q}_j}{t} + \frac{\mvec{G}(\mvec{Q}_{j+1/2}^+,\mvec{Q}_{j+1/2}^-) - \mvec{G}(\mvec{Q}_{j-1/2}^+,\mvec{Q}_{j-1/2}^-)}{\Delta x} = 0    
  \end{align*}
\end{frame}

\begin{frame}{Steps in constructing finite-volume method}
  \begin{align*}
    \pfrac{\mvec{Q}_j}{t} + \frac{\mvec{G}(\mvec{Q}_{j+1/2}^+,\mvec{Q}_{j+1/2}^-) - \mvec{G}(\mvec{Q}_{j-1/2}^+,\mvec{Q}_{j-1/2}^-)}{\Delta x} = 0    
  \end{align*}
  Hence, to completely specify a finite-volume scheme we must design
  algorithms for each of the following three steps:
  \begin{itemize}\cramplist
  \item {\bf Step 1}: A {\bf recovery scheme} (possibly with limiters)
    to compute the left/right interface values $\mvec{Q}^{\pm}$ at
    each interface using a set of cell-average values around that
    interface,%
    \mypause%
  \item {\bf Step 2}: A {\bf numerical flux function} that takes the
    left/right values and returns a consistent approximation to the
    physical flux, and%
    \mypause%
  \item {\bf Step 3}: A {\bf time-stepping scheme} to advance the
    solution in time and compute the cell-averages at the next
    time-step.
  \end{itemize}
\end{frame}  

\begin{frame}{Some notation for use in recovery stencils}
  \footnotesize%
  Example: symmetric recovery across two cells can be written as
  \begin{align*}
    Q_{i+1/2} = \frac{1}{2}(Q_{i+1}+Q_j) = \frac{1}{2}(d_p + d_m) Q_{i+1/2}
  \end{align*}
  Example: central difference scheme for second derivative:
  \begin{align*}
    \frac{\partial^2 Q_i}{\partial x^2}
    = \frac{1}{\Delta x^2} (Q_{i+1} - 2 Q_j + Q_{i-1})
    = \frac{1}{\Delta x^2} (\Delta_p - 2I + \Delta_m) Q_i
  \end{align*}      
  \begin{figure}
    \setkeys{Gin}{width=0.75\linewidth,keepaspectratio}
    \incfig{stencil-ops.png}
    \caption{Basic indexing operators to move from cell to cell, face
      to cell and cell to face.}
  \end{figure}
\end{frame}

\begin{frame}{Recovery scheme: four-cell stencil, centered scheme}
  \footnotesize%
  \begin{figure}
    \setkeys{Gin}{width=0.35\linewidth,keepaspectratio}
    \incfig{4c-stencil.png}
  \end{figure}  
  \begin{itemize}
  \item To construct a four-cell symmetric stencil recovery across an
    interface we will use a four-cell stencil:
    $\{d_{2m}, d_m, d_p, d_{2p} \}$%
    \mypause%
  \item Setup a local coordinate system with $x=0$ at the interface
    and assume a polynomial recovery
    \begin{align*}
      p(x) = p_0 + p_1 x + p_2 x^2 + p_3 x^3
    \end{align*}
    \mypause%
  \item Match the cell-averages of $p(x)$ in each of the cells
    $\{d_{2m}, d_m, d_p, d_{2p} \}$ to get a system of linear
    equations. Solve this system to determine $p_0, p_1, p_2, p_3$.
  \end{itemize}
\end{frame}

\begin{frame}{Recovery scheme: four-cell stencil, centered scheme}
  Solving the system of four equations for the four coefficients
  $p_i$, $i=0,\ldots,3$ yields:
  \begin{align*}
    p_0 &=  \frac{1}{12}(-d_{2m} + 7 d_m + 7 d_p - d_{2p}) Q \\
    p_1 &=  \frac{1}{12\Delta x}(d_{2m} - 15 d_m + 15 d_p - d_{2p}) Q \\
    p_2 &=  \frac{1}{4 \Delta x^2}(d_{2m} - d_m - d_p + d_{2p}) Q \\
    p_4 &=  \frac{1}{6 \Delta x^3}(d_{2m} - 3d_m + 3 d_p - d_{2p}) Q.
  \end{align*}
  \begin{itemize}
  \item Notice: stencils of the even coefficients are \emph{symmetric}
    and the odd coefficients are \emph{anti-symmetric}.%
    \mypause%
  \item To compute the interface value we do not really need all of
    these coefficients but only need to evaluate the recovery
    polynomial at $x=0$, i.e we only need $p(0) = p_0$
  \end{itemize}
\end{frame}

\begin{frame}{Recovery scheme: four-cell stencil, centered scheme}
  \small%
  To compute the interface value we do not really need all of these
  coefficients but only need to evaluate the recovery polynomial at
  $x=0$, i.e we only need $p(0) = p_0$. Hence, the interface value can
  be computed from
  \begin{align*}
    Q^+ = Q^- = \frac{1}{12}(-d_{2m} + 7 d_m + 7 d_p - d_{2p}) Q.
  \end{align*}
  Note that due the symmetric nature of the stencil we have only a
  \emph{single} value at the interface. This means that the numerical
  flux function at an interface is simply
  \begin{align*}
    G(Q,Q) = F(Q)
  \end{align*}
  from consistency requirements.%
  \mypause%
  This completes the spatial finite-volume discretization! The scheme
  one gets from this is very accurate (even ``structure preserving''
  for Maxwell equations), though not very robust in presence of sharp
  gradients. (No Free Lunch)
\end{frame}

\begin{frame}{How accurate is any given scheme?}
  To fix ideas consider we wish to solve the advection equation
  \begin{align*}
    \pfrac{f}{t} + \pfrac{f}{x} = 0
  \end{align*}
  Using the four-cell symmetric recovery scheme to compute interface
  values in the FV update formula we get the semi-discrete scheme
  \emph{five-cell stencil} update formula:
  \begin{align*}
    \pfrac{f_j}{t}
    = -\frac{1}{\Delta x}\int_{x_{j-1/2}}^{x_{j+1//2}}
    \pfrac{f}{x} \thinspace dx
    =
    -\frac{1}{12 \Delta x} (f_{j-2} - 8f_{j-1} + 8 f_{j+1} - f_{j+2})
  \end{align*}
  How accurate is this scheme, or what is its order of convergence?
\end{frame}

\begin{frame}{How accurate is any given scheme? Use Taylor series}
  \footnotesize%
  \begin{itemize}\cramplist
  \item Take a Taylor series polynomial around the cell center of cell
    $I_j = [-\Delta x/2, \Delta x/2]$ locally at $x=0$
    \begin{align*}
      T(x) = \sum_{n=0} \frac{T_n}{n!} x^n.
    \end{align*}
    \mypause%
  \item Compute the cell average of this polynomial in each of the
    stencil cells $\{\Delta_{2m}, \Delta_m, \Delta_p, \Delta_{2p} \}$
    \mypause%
  \item Substitute these averages in the update formula to compute the
    mean value of the flux gradient in the cell
    $I_j = [-\Delta x/2, \Delta x/2]$
    \begin{align*}
      \frac{1}{12 \Delta x} (\Delta_{2m} - 8\Delta_m + 8 \Delta_p -
      \Delta_{2p}) T
      =
      T_1 + \frac{\Delta x^2}{24} T_3  - \frac{21 \Delta x^4}{640} T_5 + \ldots
    \end{align*}
    \mypause%
  \item Subtract the exact cell average of the gradient of the Taylor
    polynomial in cell $I_j = [-\Delta x/2, \Delta x/2]$, i.e.
    \begin{align*}
      \frac{1}{\Delta x}\int_{-\Delta x/2}^{\Delta x/2} \pfrac{T}{x} \thinspace dx
      =
      T_1 + \frac{\Delta x^2}{24} T_3  + \frac{\Delta x^4}{1920} T_5 + \ldots
    \end{align*}
    from the stencil computed value. The remainder term is the error
    of the scheme.
  \end{itemize}  
\end{frame}

\begin{frame}{Symmetric four-cell recovery scheme is fourth-order
    accurate}
  The above procedure (needs use of a compute algebra system to
  simplify the computations) shows that the symmetric four-cell
  recovery scheme has error that goes like
  \begin{align*}
      \frac{\Delta x^4}{30} T_5 + O(\Delta x^6)
  \end{align*}
  showing the scheme converges with \emph{fourth-order} accuracy
  $O(\Delta x^4)$ for linear advection equation. (Reducing $\Delta x$
  by 2 reduces error by a factor of $16$).

\end{frame}

\begin{frame}{Accuracy is not everything: dispersion and diffusion}
  \footnotesize
  \begin{itemize}
  \item High-order symmetric schemes like the one we derived are very
    accurate (even ``structure preserving'' for some problems) but not
    robust.
    \mypause%
  \item Two other properties of the scheme are important to
    understand: \emph{dispersion} and \emph{diffusion}. For this we
    wil derive a \emph{numerical dispersion relation} analogous to
    dispersion relation we derived for linearized systems.
    \mypause%
  \item Consider a single mode $f(x) = e^{ik x}$ where $k$ is the
    wavenumber. Compute the cell-average of the mode on each of the
    cells in the stencil, plug into the stencil formula to derive the
    \emph{numerical dispersion relation}
    \begin{align*}
      i\overline{k} = \sum_{m=-N}^{M} c_m e^{i k m \Delta x}
    \end{align*}
    where we have written the stencil in the generic form
    \begin{align*}
      \frac{1}{\Delta x}\sum_{m = -N}^{M} c_m f_{j+m}
    \end{align*}

  \end{itemize}
\end{frame}

\begin{frame}{Symmetric four-cell recovery scheme has no diffusion!}
  \small
  \begin{columns}
    
    \begin{column}{0.6\linewidth}
      \begin{itemize}
      \item Note that the numerical dispersion relation will in
        general give a \emph{complex} effective wavenumber
        $\overline{k}$.
      \item The dispersion relation for a hyperbolic equation is
        $\omega = \lambda k$. Hence, the \emph{real part} of
        $\overline{k}$ represents dispersion and \emph{imaginary part}
        of $\overline{k}$ represents diffusion/growth. Obviously, we
        want imaginary part to be \emph{negative} to avoid solution
        blow-up!
      \item The four-cell symmetric stencil has \emph{no imaginary
          part} of $\overline{k}$. This related to the fact that it is
        \emph{symmetric} (anti-symmetric stencil coefficients). This
        is not necessarily a good thing!
      \end{itemize}
    \end{column}
  
    \begin{column}{0.4\linewidth}
      \begin{figure}
        \includegraphics[width=\linewidth]{5p-kbar.png}
        \caption{Real-part of numerical dispersion relation for
          four-cell recovery scheme. Notice the strong dispersion for
          higher-$k$ modes}
      \end{figure}
    \end{column}
  \end{columns}

\end{frame}  

\end{document}


\begin{frame}{}
\end{frame}

\begin{columns}
  
  \begin{column}{0.6\linewidth}
  \end{column}
  
  \begin{column}{0.4\linewidth}
    \includegraphics[width=\linewidth]{fig/Kinsey_2011_Pfus_vs_T.pdf}
  \end{column}
\end{columns}

% ----------------------------------------------------------------
