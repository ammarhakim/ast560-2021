\documentclass[aspectratio=169]{beamer}
%\documentclass[aspectratio=43]{beamer}

\usepackage{graphicx}  % Required for including images
\usepackage{natbib}
\usepackage{booktabs} % Top and bottom rules for tables
\usepackage{amssymb,amsthm,amsmath}
\usepackage{exscale}
\usepackage{natbib}
\usepackage{tikz}
\usepackage{listings}
\usepackage{color}
\usepackage{bm}
% Setup TikZ
\usepackage{tikz}
\usetikzlibrary{arrows}
\tikzstyle{block}=[draw opacity=0.7,line width=1.4cm]
% Setup hyperref
\usepackage{hyperref}
\hypersetup{colorlinks=true}
\hypersetup{citecolor=porange}
\hypersetup{urlcolor=porange!80!}
\hypersetup{linkcolor=porange}

\newtheorem{proposition}{Proposition}
\newtheorem{remark}{Remark}
\newtheorem{principle}{Principle}

%% Writing quarters
\newcommand{\wQ}[1]{{\textcolor{white}{Q#1}}}
\newcommand{\bQ}[1]{{Q#1}}

% Uncomment appropriate command to disable/enable hiding
%\newcommand{\mypause}{\pause}
\newcommand{\mypause}{}
\newcommand{\myb}[1]{{\color{blue} {#1}}}

%% Commonly used macros
\newcommand{\eqr}[1]{Eq.\thinspace(#1)}
\newcommand{\pfrac}[2]{\frac{\partial #1}{\partial #2}}
\newcommand{\pfracc}[2]{\frac{\partial^2 #1}{\partial #2^2}}
\newcommand{\pfraca}[1]{\frac{\partial}{\partial #1}}
\newcommand{\pfracb}[2]{\partial #1/\partial #2}
\newcommand{\pfracbb}[2]{\partial^2 #1/\partial #2^2}
\newcommand{\spfrac}[2]{{\partial_{#1}} {#2}}
\newcommand{\mvec}[1]{\mathbf{#1}}
\newcommand{\gvec}[1]{\boldsymbol{#1}}
\newcommand{\script}[1]{\mathpzc{#1}}
\newcommand{\eep}{\mvec{e}_\phi}
\newcommand{\eer}{\mvec{e}_r}
\newcommand{\eez}{\mvec{e}_z}
\newcommand{\iprod}[2]{\langle{#1}\rangle_{#2}}

\DeclareMathAlphabet{\mathpzc}{OT1}{pzc}{m}{it}

%% Autoscaled figures
\newcommand{\incfig}{\centering\includegraphics}
\setkeys{Gin}{width=0.9\linewidth,keepaspectratio}

%Make the items smaller
\newcommand{\cramplist}{
	\setlength{\itemsep}{0in}
	\setlength{\partopsep}{0in}
	\setlength{\topsep}{0in}}
\newcommand{\cramp}{\setlength{\parskip}{.5\parskip}}
\newcommand{\zapspace}{\topsep=0pt\partopsep=0pt\itemsep=0pt\parskip=0pt}

\newcommand{\backupbegin}{
   \newcounter{finalframe}
   \setcounter{finalframe}{\value{framenumber}}
}
\newcommand{\backupend}{
   \setcounter{framenumber}{\value{finalframe}}
}

\usetheme[bullet=circle,% Use circles instead of squares for bullets.
          titleline=true,% Show a line below the frame title.
          ]{Princeton}

\title[{\tt }]{Hyperbolic PDEs and Finite-Volume Methods I}%
\author[https://ast560.rtfd.io]%
{Ammar H. Hakim ({\tt ammar@princeton.edu}) \inst{1}}%

\institute[PPPL]
{ \inst{1} Princeton Plasma Physics Laboratory, Princeton, NJ %
}

\date[2/9/2021]{Princeton University, Course AST560, Spring 2021}

\begin{document}

\begin{frame}[plain]
  \titlepage
\end{frame}

%----------------------------------------------------------------
\begin{frame}{Goal: Methods for solution of hyperbolic/mixed PDEs}

  Hyperbolic (or mixed) PDEs appear everywhere in physics and
  engineering: Euler equations, Navier-Stokes equations, MHD
  equations, Einstein's equation of general relativity, shallow-water
  equations, ....%
  \mypause%
  \begin{itemize}
  \item Describe phenomena that travel with \emph{finite-speed}:
    clearly all fundamental physics must obey causality, hence
    hyperbolic systems are fundamental (though approximations of
    hyperbolic equations may violate finite-speed constraint).
    \mypause%
  \item Display very rich structure shocks and other discontinuities,
    instabilities, turbulence ... and in most problems, a mix of all!
    \mypause%
  \item Specialized methods are needed to solve such systems: naive
    algorithms can {\bf cause disaster}!
  \end{itemize}
  \mypause%
  My goal is to develop \emph{conceptual} understanding about such
  equations and numerical methods to solve them. Literature is too
  vast to cover in few lectures!

\end{frame}

% ----------------------------------------------------------------

\begin{frame}{No Free Lunch Principle}

  \begin{block}{No Free Lunch Principle}
    There is no \emph{unique} discrete system of equations
    corresponding to a given system of continuous equations. No
    discrete system is perfect and a method that works well in one
    situation may not work well in others.
  \end{block}
  \vskip0.1in%
  \mypause%
  ``All numerical methods suck, though some suck less than
  others. Make sure your method sucks less that your competition''
  
\end{frame}

% ----------------------------------------------------------------

\begin{frame}{Dissipation, dispersion and robustness}
  Typically, numerical methods trade between accuracy and
  robustness. Very accurate schemes are not too robust and highly
  robust schemes are typically not very accurate. Must find a balance.
  \mypause%
  \begin{itemize}
  \item To capture shocks and sharp gradient features some localized
    diffusion (dissipation) is needed via \emph{limiters} or some
    other means. Some dissipation may also be needed for stability.
  \item Dissipation typical leads to a loss of other properties: in
    particular, for Maxwell equations it can cause EM energy to
    \emph{decay}. For Euler equations high-k modes may get overdamped
    due to limiters: not good for turbulence problems.%
    \mypause%
  \item An ideal situation is to apply dissipation only where it is
    needed and use low-dissipation schemes elsewhere. However, this is
    easier said than done. How to determine where to apply dissipation
    is very tricky, specially in nonlinear complex flows. Ease to
    confuse physical features for numerical artifacts (and
    vice-versa!)
  \end{itemize}
\end{frame}

\begin{frame}{Hyperbolic PDEs describe phenomena that travel at finite
    speed}
  \footnotesize%
  An intutive ``definition'' that we will initially work with before
  stating the mathematically rigorous definition:
  \begin{definition}[Hyperbolic PDEs ``Intuitive Definition'']
    A hyperbolic PDE is one in which all phenomena travel at a finite
    speed.
  \end{definition}

  \mypause%
  \vskip0.1in%
  Some prototypical examples we will look at more closely:
  \begin{itemize}
  \item Advection equation: simplest trivial \emph{linear} hyperbolic
    equation. Trival but very important!
  \item Maxwell equation of electromagnetism: linear hyperbolic system
  \item Euler equations: probably historically the most important
    \emph{nonlinear hyperbolic system}. Basis of vast literature on
    numerical methods and basis for more complex equations: ideal MHD,
    (general) relativistic hydro/MHD, Navier-Stokes solvers, etc. Need
    to understand even if you want to follow literature and apply
    methods to your own problem.
  \end{itemize}
  
\end{frame}

\begin{frame}{Hyperbolic PDEs: rigorous definition, no reliance on
    linearization}
  Consider a system of conservation laws written as
  \begin{align*}
    \pfrac{\mvec{Q}}{t} + \pfrac{\mvec{F}}{x} = 0.
  \end{align*}
  where $\mvec{Q}$ is a vector of conserved quantities and
  $\mvec{F}(\mvec{Q})$ is a vector of fluxes. This system is called
  \emph{hyperbolic} if the flux Jacobian
  \begin{align*}
    \mvec{A} \equiv \pfrac{\mvec{F}}{\mvec{Q}}
  \end{align*}
  has \emph{real eigenvalues} and a \emph{complete set of linearly
    independent} eigenvectors. In multiple dimensions if $\mvec{F}_i$
  are fluxes in direction $i$ then we need to show that arbitrary
  linear combinations
  $\sum_i n_i {\partial\mvec{F}_i}/{\partial\mvec{Q}}$ have real
  eigenvalues and linearly independent set of eigenvectors.
  
\end{frame}

\begin{frame}{To compute eigensystem often easier to work in
    \emph{quasilinear form}}
  To derive eigensystem it is sometimes easier to work in
  non-conservative (quasi-linear) form of equations. Start with
  \begin{align*}
    \pfrac{\mvec{Q}}{t} + \pfrac{\mvec{F}}{x} = 0.
  \end{align*}
  and introduce an invertible transform $\mvec{Q} = \varphi(\mvec{V})$
  where $\mvec{V}$ are some other variables (for example: density,
  velocity and pressure). Then the system converts to
  \begin{align*}
    \pfrac{\mvec{V}}{t} +
    \underbrace{(\varphi^{\prime})^{-1} \mvec{A}\varphi^{\prime}}_{\mvec{B}}
    \pfrac{\mvec{V}}{x} = 0.
  \end{align*}
  Can easily show eigenvalues of $\mvec{A}$ are same as that of
  $\mvec{B}$ and right eigenvectors can be computed from
  $\varphi^{\prime} \mvec{r}_p$ and left eigenvectors from
  $\mvec{l}_p (\varphi^{\prime})^{-1}$.
\end{frame}

\begin{frame}{Example 1: Euler equations of invicid fluids}
  \begin{align*}
    \frac{\partial}{\partial{t}}    
    \left[
    \begin{matrix}
      \rho \\
      \rho u \\
      \rho v \\
      \rho w \\
      E
    \end{matrix}
    \right]
    +
    \frac{\partial}{\partial{x}}
    \left[
    \begin{matrix}
      \rho u \\
      \rho u^2 + p \\
      \rho uv \\
      \rho uw \\
      (E+p)u
    \end{matrix}
    \right]
    =
    0    
  \end{align*}
  Here $E = p/(\gamma-1) + \rho u^2/2$ is the total
  energy. Eigenvalues of this system are $\{u-c_s,u,u,u,u+c_s \}$
  where $c_s = \sqrt{\gamma p/rho}$ is the sound speed. See class
  notes for left/right eigenvectors.%
  \vskip0.1in%
  Note: in the limit $p\rightarrow 0$ all eigenvalues become $u$ and
  for cold-fluid ($p=0$) the system does not possess complete set of
  eigenvectors. (Cold fluid model is important to model dust, for
  example, in astrophysical systems or in say volcanic explosions).
\end{frame}

\begin{frame}{Example 2: Ideal MHD equations}
  \footnotesize%
  Ideal MHD equations are very important to both fusion and
  astrophysical problems. Written in non-conservative form they are
  \begin{align*}
    \frac{\partial \rho}{\partial t}+ \mvec{u}\cdot\nabla\rho + \rho\nabla\cdot\mathbf{u}&=0 \\
    \frac{\partial \mathbf{u}}{\partial t}+\mathbf{u} \cdot \nabla
    \mathbf{u}
    +\frac{\nabla p}{\rho} &=
                             \color{blue}{\frac{1}{\mu_{0}\rho}(\nabla \times \mathbf{B}) \times \mathbf{B}} \\
    \frac{\partial p}{\partial t}+\mathbf{u} \cdot \nabla p+\gamma p
    \nabla \cdot \mathbf{u} &= 0 \\
    \color{blue}{\frac{\partial \mathbf{B}}{\partial t}-\nabla \times(\mathbf{u} \times \mathbf{B})} &= \color{blue}{0}
  \end{align*}
  with the constraint $\nabla\cdot\mvec{B} = 0$. The eigensystem is
  complicated to compute! (Try doing it yourself). Eigenvalues are
  $u\pm c_f$, $u\pm c_s$, $u\pm c_a$, and $u$ (7 eigenvalues for 8
  equations). Here $c_f$, $c_s$ are the fast/slow magnetosonic speeds
  and $c_a$ is the Alfven speed. See Ryu and Jones ApJ {\bf 442}
  228-258, 1995 (linked on website).
\end{frame}

\end{document}


\begin{frame}{}
\end{frame}

\begin{columns}
  
  \begin{column}{0.6\linewidth}
  \end{column}
  
  \begin{column}{0.4\linewidth}
    \includegraphics[width=\linewidth]{fig/Kinsey_2011_Pfus_vs_T.pdf}
  \end{column}
\end{columns}

% ----------------------------------------------------------------
